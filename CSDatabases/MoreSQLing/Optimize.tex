\documentclass[11pt]{article}
\renewcommand{\baselinestretch}{1.05}
\usepackage{amsmath,amsthm,verbatim,amssymb,amsfonts,amscd, graphicx}
\usepackage{graphics}
\topmargin0.0cm
\headheight0.0cm
\headsep0.0cm
\oddsidemargin0.0cm
\textheight23.0cm
\textwidth16.5cm
\footskip1.0cm
\theoremstyle{plain}
\newtheorem{theorem}{Theorem}
\newtheorem{corollary}{Corollary}
\newtheorem{lemma}{Lemma}
\newtheorem{proposition}{Proposition}
\newtheorem*{surfacecor}{Corollary 1}
\newtheorem{conjecture}{Conjecture} 
\newtheorem{question}{Question} 
\theoremstyle{definition}
\newtheorem{definition}{Definition}

 \begin{document}
 


\title{MoreSQL-ing}
\author{Jeremy Desser}
\begin{titlepage}
\maketitle
\end{titlepage}


\section{Query 1} 

\subsection{}This query returned an error when ran. Therefore the cost of initially running it was Null.
\subsection{}
To get this query to return anything, I made the following changes:\\
SELECT \\
    DISTINCT m.nameFirst, m.nameLast, a.W\\
FROM\\
    master m,\\
    pitching a,\\
    teams t\\
WHERE
    m.masterId = a.masterId\\
        AND t.teamId = a.teamId\\
        AND t.lgID = a.lgID\\
        AND t.name like\\
		"Montreal Expos"\\
AND t.yearID = a.yearId AND a.w {\geq 20}\\


\subsection{}
After the above changes, it took .141 seconds to return the answer.

\section{Query 2}
\subsection{} Like the above query, running this question returned no tuples for me. After switching the given query from 3 queries joined together, to one query with multiple JOIN statements, I got an answer returned.
\subsection{}
SELECT h / ab as Average, h as "Hits", ab as "At Bats", nameFirst as "First Name",\\ nameLast as "Last Name", batting.yearID as Year\\
FROM  master as m JOIN batting ON m.masterID = batting.masterID \\
	JOIN schoolsplayers AS s ON m.masterID = s.masterID \\
	JOIN schools as c ON c.schoolID = s.schoolID\\
WHERE c.schoolName = 'Utah State University' && batting.H \geq 0\\
ORDER BY batting.yearID ASC ;\\

I changed the multiple query join to a join with schoolsplayers and schools and it returned rows.\\
\subsection{}
After the above changes, it took 0.094 seconds to run.
\section{Query 3}
\subsection{}The given query wouldn't finish running on my machine(older machine).
\subsection{}
It was difficult to see what the query was trying to do. If you want to find the people Derek Jeter played with on a team you can do a query to find all the years and teams Derek Jeter played for, then INNER JOIN that with a list of all the players.\\

SELECT DISTINCT players.yearID as "Year", players.nameFirst as "First Name",\\ players.nameLast as "Last Name"\\
    FROM (\\
    SELECT yearId, name AS teamName, nameFirst, nameLast\\
	FROM master\\
	NATURAL JOIN appearances\\
	NATURAL JOIN teams) as players INNER JOIN (\\
    SELECT yearId, nameFirst, nameLast, name AS teamName\\
FROM master\\
	NATURAL JOIN teams\\
	NATURAL JOIN appearances\\
WHERE nameFirst = "Derek"\\
	AND nameLast = "Jeter") as Jeter\\
    ON players.yearID = Jeter.yearId AND players.teamName = Jeter.teamName\\
\subsection{}
Such a query took my machine 0.938 seconds to run, which was better than the first one which crashed mySQL workbench and never finished running.
\clearpage
\section{Query 4}
\subsection{}This query took .094 seconds to run.
\subsection{}This query was pretty fast to begin with and I could not think of any ways to improve upon it.\\
You need the subquery to find the max wins of each year, then the main query to show each of the highest wins of the year played.
\subsection{} N/A

\section{Query 5}
\subsection{} This query took 40.76 seconds to run.
\subsection{}Although I had problems optimizing this one you could create a view to find all the batting averages. Then find the averages of those averages and return which averages are bigger than the average of all the batters.
\subsection{} Couldn't get the query to run.

\section{Query 6}
\subsection{}
This query took 3.68 seconds to run on my computer.
\subsection{}
This query would be rough to optimize since to find the 4 years you need to join the 4 queries. Perhaps adding more constraints on the end could speed it up. I personally couldn't get it much faster
\subsection{}
Trying to optimize this query I got one that was about 3.375 seconds,but I had left out a query on accident and one that was far slower than the original at 17 seconds.
\section{Query 7}
\subsection{}
This query ran in .218 seconds.
\subsection{}This query was fast to begin with. I couldn't get it to run much faster. It is necessary to run the sub-queries to get the consecutive years and the first query to find the team names needed.
\subsection{}N/A

 
 
 
\end{document}