\documentclass[11pt]{article}
\renewcommand{\baselinestretch}{1.05}
\usepackage{amsmath,amsthm,verbatim,amssymb,amsfonts,amscd, graphicx}
\usepackage{graphics}
\topmargin0.0cm
\headheight0.0cm
\headsep0.0cm
\oddsidemargin0.0cm
\textheight23.0cm
\textwidth16.5cm
\footskip1.0cm
\theoremstyle{plain}
\newtheorem{theorem}{Theorem}
\newtheorem{corollary}{Corollary}
\newtheorem{lemma}{Lemma}
\newtheorem{proposition}{Proposition}
\newtheorem*{surfacecor}{Corollary 1}
\newtheorem{conjecture}{Conjecture} 
\newtheorem{question}{Question} 
\theoremstyle{definition}
\newtheorem{definition}{Definition}
\usepackage{graphicx}
\graphicspath{ {images/} }
 \begin{document}
 


\title{ER Diagram}
\author{Jeremy Desser}
\begin{titlepage}
\maketitle
\end{titlepage}


\section{}
\includegraphics[scale=.9]{ErDiagram}
\section{Overview}
For my ER Model I chose to do field biology in the form of a conservationist effort. There are a few things needed for such a model. The conservation, and the animals are the most important things. The animals have a diet, a habitat, and a tracking device with a unique ID. The conservation need employees, a company or non-profit associated with it, and a location around the world they are working from. Each employee also has a one to many jobs it performs. This can be any number of things such as local veterinarian, caretaker, ect. 


\section{Entity Types}
Habitat$(\underline{country }, Biome, IsGovermentProtected,\underline{ longitude_atitude},\underline{ Animals_idAnimals})$ \\

Diet$(\underline{ idDiet}, Herbavore, Carnavore, SpecificFood, foodLocal)$ \\ 

Animals$(\underline{ animalType}, \underline{ idAnimals}, weight, endangered, \underline{diet_idDiet}, \underline{ trackingID})$ \\

Tracking$(\underline{ trackingNumber}, initalGpsPosition, lastKnownGpsPosition)$ \\

Employee$(\underline{ idEmployee}, firstName, lastName, countryStationed)$ \\

Jobs$(\underline{ IdJob}, JobTitle, JobDiscription, EmployeeId)$ \\ \\
Most keys were pretty simple habitat key is a combination of physical location on the globe and a country name since that will always be unique. Diet and animal have a foreign key of animalID and animalType. Animal has a combination of ID and animal type for its keys. Tracking makes sense as a unique number since that is what it would be in the wild.
Each employee would have an ID, and personal information like a first and last name, SSN that could be used as a key.

\section{Relationship Types}
Conservationist Company $(\underline{ conservationist_idConservationist}, \underline{ employee_idEmployee}, employeeName)$ \\

Animals\_under\_protection$(\underline{Conservationist_idConservationist},\underline{Animals_idAnimals}, companyName, AnimalType)$\\ 

\section{Alpha Contributions}
1. An Animal has a Diet. it can eat one to many things, and the animal has total participation.\\
2. An animal may live in many different locations, and a habitat might house many different animals.\\
3. An animal has a diet, a habitat, and may have a place in which its under protection\\
4. An animal might have protection of a conservation, or need protection of a conservation.\\
5. An employee has a job or many jobs.\\
6. The diet table has many values that are only boolean.\\
7. Since an employee can have multiple jobs, this would be multi-valued.\\

\section{Beta Contributions} 
I did the project solo, so I did all the contributions myself.
 
\end{document}