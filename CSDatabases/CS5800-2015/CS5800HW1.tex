\documentclass[11pt]{article}
\renewcommand{\baselinestretch}{1.05}
\usepackage{amsmath,amsthm,verbatim,amssymb,amsfonts,amscd, graphicx}
\usepackage{graphics}
\topmargin0.0cm
\headheight0.0cm
\headsep0.0cm
\oddsidemargin0.0cm
\textheight23.0cm
\textwidth16.5cm
\footskip1.0cm
\theoremstyle{plain}
\newtheorem{theorem}{Theorem}
\newtheorem{corollary}{Corollary}
\newtheorem{lemma}{Lemma}
\newtheorem{proposition}{Proposition}
\newtheorem*{surfacecor}{Corollary 1}
\newtheorem{conjecture}{Conjecture} 
\newtheorem{question}{Question} 
\theoremstyle{definition}
\newtheorem{definition}{Definition}

 \begin{document}
 


\title{Homework 1\\CS5800}
\author{Jeremy Desser}
\begin{titlepage}
\maketitle
\end{titlepage}


\section{Relational Algebra}
\subsection{}
\sigma_{AGE } Actor
\subsection{}
\pi_{Title}(\sigma_{WhenReleased = "1940" } Movie)
\subsection{}
1. \pi_{Title}(\sigma_{Cost>1000000 \wedge WhenReleased<"1920"}Movie) \\
2. Movies$_{Title }$ \cap Movies$_{cost > 1000000 \wedge WhenReleased < "1920"}$
\subsection{}
 \pi_{Name, Age}(\sigma_{Movie.WhenReleased =" 1940"}(Actor \bowtie (CastIn \bowtie Movie)))
\subsection{}
\pi_{Name, Age}(\sigma_{Movie.WhenReleased <" 1920"}(Actor \bowtie (CastIn \bowtie Movie)))
\subsection{}
(\sigma_{Name} Actor) - (\pi_{Actor.Name}(\sigma CastIn.Name(Actor \bowtie (CastIn \bowtie Movie)))
\subsection{}
\pi_{Name}(\sigma_{$R_1$.Name = $R_2$.Name}(\rho($R_1$,(Actor \bowtie CastIn))X \rho($R_2$,(Actor \bowtie CastIn))))
\subsection{}
\pi_{Name}( (CastIn \bowtie (Movie \bowtie Actor) )
\subsection{}
\rho $R_1$(Actor \bowtie CastIn \: \bowie \sigma_{WhenReleased} \bowtie Movie)\\
\rho $R_2$(Actor \bowtie CastIn \: \bowie \sigma_{WhenReleased} \bowtie Movie)\\
\rho$R_3$(Actor \bowtie CastIn \: \bowie \sigma_{WhenReleased} \bowtie Movie))\\
\pi_{Name}(\sigma_{$R_1$.WhenReleased = $R_2$.WhenReleased+1 \wedge $R_2$.WhenReleased = $R_3$+1 \vee $R_1$.WhenReleased = $R_2$.WhenReleased-1 \wedge $R_2$.WhenReleased = $R_3$.WhenReleased+1 }($R_1$ \bowtie ($R_2$ \bowtie $R_3$)))
\subsection{}
(\pi_{Name}(Actor \bowtie \:(Movie \bowtie CastIn )))\bowtie(\pi_{Name}(\sigma_{$R_1$.Name = $R_2$.Name}(\rho($R_1$,(Actor \bowtie CastIn))X \rho($R_2$,(Actor \bowtie CastIn)))))-(\sigma_{Name} Actor) - (\pi_{Actor.Name}(\sigma CastIn.Name(Actor \bowtie (CastIn \bowtie Movie)))
\section{Exercise 2.14}
Three-tier architecture, it adds a middle layer that helps keep all the rules managed for the clients.
The two-tiered architecture might end up being a bit confusing for someone just looking to buy plane tickets since they would have to deal with api's and such.
\section{Exercise 3.12}
a. update Insert and Number\_of\_available\_seats of Leg\_Instance with flight\_number = xxxxx to Number\_of\_available\_seats = Number\_of\_available\_seats -1.\\
b. You would need to check that there are no seats left, the flight date could have passed\\
c.The Insert Operation onto the Seat\_reservation table will check all the constraints for the relation.
number\_of\_available\_seats on each Leg\_Instance of the flight \textless \: 1 doesn't fall into any of the listed constraints.\\

\section{Exercise 3.14}
 Order(Cust\#) is a foreign key on Customer(Cust\#) \\
 Order(Order\#) is a foreign key on Order\_Item(Order\#)\\
 Shipment(Warehouse\#)  is a foreign key on Warehouse(Warehouse\#)\\
  Order(Order\#)  is a foreign key on Shipment(Order\#)

 
\end{document}